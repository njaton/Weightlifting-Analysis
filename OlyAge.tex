\documentclass{article}
\usepackage{graphicx}
\graphicspath{ {./ageGraphs/} }

%Create Title
\title{Age Analysis in Olympic Weightlifters}
\author{Nick Jaton}
\date{August 2018}

\begin{document}

%build title
\begin{titlepage}
\maketitle
\end{titlepage}

%Introduction 
\section{Introduction}
	This is a quick dive into the age range of olympic athletes, specifically weightlifters. All data used in this study was acquired from Randi H Griffin's "120 years of Olympic history: athletes and results" which is available on Kaggle. Upon starting this project it was hypothesized that the age range of both male and female athletes would be in the higher 20's. Though this data did not yield dramatically different results it seems that the athletes are more competitive at younger ages than expected. Please feel free to download the R code to get a better understanding of the experiment.

\section{Methods}
	The methods used for this project are very straightforward. In order to easily analyze the data, the data regarding weightlifters was extracted. After this the data was separated into different data sets by either medal or sex. All that was left to do was looking into the average age of these medalists and compare them to athletes that did not earn a medal.

\section{Results}
	The average age of athletes that earned gold medals was younger than hypothesized. 24.59 years of age was the average age for gold medalists of both sexes. Generally speaking male athletes tended to be around 1.5 - 2 years older on average than female athletes. This did not seem to change depending on the athletes medal earning status. Unexpectedly, the average age of medalist was not much lower than other non-medalists. This is made clear in Figures 9 and 10.

\section{Conclusion}
	This project has lead to just as many questions as answers. The data suggests that age did not have a dramatic affect on an athlete's ability to earn a medal in the olympics but, it is not enough to assume this from the current information collected. This study would stand to benefit from breaking down various age ranges to see how competetive each age range is at earning medals. When looking through the various figures created it can be seen that there are far more athletes in their mid-twenties than otherwise.  

\newpage
%Figures
\section{Figures}

\begin{figure}[ht!]
\subsection{Figure: 1}
\centering
\includegraphics[scale=.7]{Rplot1}
\end{figure}

\begin{figure}[ht!]
\subsection{Figure: 2}
\centering
\includegraphics[scale=.7]{Rplot2}
\end{figure}

\begin{figure}[ht!]
\subsection{Figure: 3}
\centering
\includegraphics[scale=.7]{Rplot3}
\end{figure}

\begin{figure}[ht!]
\subsection{Figure: 4}
\centering
\includegraphics[scale=.7]{Rplot4}
\end{figure}

\begin{figure}[ht!]
\subsection{Figure: 5}
\centering
\includegraphics[scale=.7]{Rplot5}
\end{figure}

\begin{figure}[ht!]
\subsection{Figure: 6}
\centering
\includegraphics[scale=.7]{Rplot6}
\end{figure}

\begin{figure}[ht!]
\subsection{Figure: 7}
\centering
\includegraphics[scale=.7]{Rplot7}
\end{figure}

\begin{figure}[ht!]
\subsection{Figure: 8}
\centering
\includegraphics[scale=.7]{Rplot8}
\end{figure}

\begin{figure}[ht!]
\subsection{Figure: 9}
\centering
\includegraphics[scale=.7]{Rplot9}
\end{figure}

\begin{figure}[ht!]
\subsection{Figure: 10}
\centering
\includegraphics[scale=.7]{Rplot10}
\end{figure}

\end{document}
