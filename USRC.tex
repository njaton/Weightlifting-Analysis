\documentclass{article}
\usepackage{graphicx}
\usepackage{amsmath}
\graphicspath{ {./USRCGraphics/} }

%Create Title
\title{US vs Russia vs China in the Sport of Olympic Weightlifting}
\author{Nick Jaton}
\date{August 2018}

\begin{document}

%Build title
\begin{titlepage}
\maketitle
\end{titlepage}

%Introduction
\section{Introduction}
	Russia and China are two countries that are known to be strong competitors in the weightlifting community. This document will compare both of these teams as well as my home country, The United States of America. The focus of this comparison will be looking into how many medal earning athletes are produced by each country. Please feel free to download the R code to get a better understanding of the methods used.

\section{Methods}
	As mentioned in the introduction R was used for all aspects of this project. The data for this project was provided by Randi H Griffen on Kaggle. The title for this data set is ”120 years of Olympic history: athletes and results”. Each team was split up by sex and the medal that was earned by the athlete. Much of the data from this set was then converted to percentages to better compare the teams.

\section{Results}
	To shine some light on the data before going into more details regarding the team's ability to earn medals, the amount of athletes that competed for each team was determined in Figure: 1. Since the beginning of the Olympic games China has only had 73 male athletes compete, which is only 52.51\% as much as the United States 139 athletes. Russia on the other hand has had even fewer athletes at 31. Unlike male athletes, female athlete numbers seem to be much closer to each other. Russia has had the smallest amount of athletes make it to the Olympics with 14 athletes and China has had the most at 20 female lifters compete. The United States had 15 female athletes, only one more than russia. 
\bigbreak
	Even though China has not had as many athletes compete in the Olympics they have earned the same amount of male weightlifting medals and far more female medals than the United States. Both China and the United States had 39 male athletes earn a medal. That means that means that 52.4\% of male Chinese athletes placed in the top three of their class. Russia ended up with a very similar number with 51.6\% of their male athletes placing in the top three. The United States has had only 28.06\% of their male athletes land in the top three of their perspective class. The most unexpected part of this data was the female chinese athletes. 90\% of female chinese athletes took home a medal. On average 71.43\% of female Russian athletes earned a medal and 20\% of female US athletes made it to the top three spots. Figure: 2 shows the amount of medals earned by each team.
\bigbreak
	After determining the amount of medals earned by each team, it seemed worthwhile to see how many of each of these athletes landed in Gold, Silver and Bronze. At this point it makes sense that 23.49\% of Chinese male lifters earned a gold medal or that 10.79\% of American male lifters earned gold medals. This data can be seen in Figure: 3. What may still be surprising is that 85\% of female lifters from china earned a gold medal.  Only 3 female lifters from china since the 2000 olympics did not earn a gold medal. Russia also had a large amount of female lifters earning silver medals at 50\%. Figure: 4 shows the effectiveness of each female team.

\section{Conclusion}
	As this project progressed it became more clear how dominant China's weightlifting team is. The impressive female lifters made this very clear. Russia also is shown to produce competitive athletes with over 50\% of their male athletes and 50\% of their female athletes taking home medals. Next to China and Russia the United States is a far less competitive team. This is especially true in regards to female athletes where only 6.8\% of athletes took home medals.

%Figures Section
\newpage
\section{Figures}
\begin{figure}[ht!]
\centering
\subsection{Figure: 1}
\centering
\includegraphics[scale=.5]{ComparingAthleteQuantitiesByTeam}
\end{figure}

\begin{figure}[ht!]
\centering
\subsection{Figure: 2}
\centering
\includegraphics[scale=.5]{ComparingQuantitiesOfWinningAthletesByTeam}
\end{figure}

\begin{figure}[ht!]
\centering
\subsection{Figure: 3}
\centering
\includegraphics[scale=.6]{PercentOfWinningAthletesByTeamMale}
\end{figure}

\begin{figure}[ht!]
\centering
\subsection{Figure: 4}
\centering
\includegraphics[scale=.6]{PercentOfWinningAthletesByTeamFemale}
\end{figure}

\end{document}
